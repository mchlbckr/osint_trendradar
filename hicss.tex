% File hicss51.tex
%%
%% Based on the style files for ACL 2015 by 
%% car@ir.hit.edu.cn, gdzhou@suda.edu.cn


\documentclass[10pt]{article}
\usepackage[letterpaper]{geometry}
\usepackage{hicss51}
\usepackage{times}
\usepackage[none]{hyphenat}
\usepackage{url}
\usepackage{latexsym}
%\usepackage{minted}
\usepackage{indentfirst}
\usepackage{graphicx}
%\graphicspath{{images/}}
\usepackage{wrapfig}
\usepackage{todonotes}
\usepackage{hyperref}
\usepackage[utf8]{inputenc}
\newcommand{\sansserifformat}[1]{\fontfamily{cmss}{ #1}}%

%\setlength\titlebox{5cm}



% You can expand the titlebox if you need extra space
% to show all the authors. Please do not make the titlebox
% smaller than 5cm (the original size).


\title{Please Read Carefully Detailed Formatting Guidelines for Preparing Your HICSS Final Paper with Author Names}

\author{Franz Kayser \\
  ESG \\
  {\underline{ franz.kayser@esg.de}} \\\And
  Thomas Mayer \\
  ESG  \\
  {\underline{ thomas3.mayer@esg.de} }\\\And 
  Michael Bücker \\
  FH Münster -- University of Applied Sciences\\
  {\underline{michael.buecker@fh-muenster.de}} \\}

\date{}

\begin{document}
\maketitle
\begin{abstract}
    Open Source Intelligence (OSINT) is currently experiencing an unprecedentedly intensive discourse, notably heightened since the Russian invasion of Ukraine. Despite numerous attempts at standardized definitions and frameworks, the intelligence discipline remains ambiguous. This paper introduces a practice-validated OSINT trend radar, categorizing technologies by maturity, intelligence cycle phase, and use case. Serving as a profound knowledge base and tool for identifying research gaps, the radar emerges from a structured design process. Sixty studies underwent categorization and validation through expert interviews, revealing the absence of a comprehensive, automated third-generation OSINTsystem in Germany. Technological gaps, especially in the planning, direction, dissemination, and integration phases, are evident. Although intelligent support technologies were identified, practical implementation lags behind theory. The human factor therefore remains central to the OSINT process. Future research should thus prioritize developing applications for underserved phases, probing reasons for limited widespread implementation of proven applications, with emphasis on legal, ethical, political, and social parameters.


\end{abstract}

\section{Introduction}

OSINT is a currently more debated research field than ever before. Obtaining intelligence from publicly available data [1] has become undeniably important since the Russian invasion of Ukraine in 2022. In this context, real-time analysis, especially of social media, has revealed highly relevant insights [2,3]. However, OSINT itself is not a new technique [4,5], but one of the oldest intelligence disciplines [6]. Despite numerous attempts to define OSINT [vgl. z.B.: 7–9], the concept of intelligent analysis remains controversial to this day [10–12]. This is not the least because every definition of OSINT is subject to advances in computer and data science, which are continuously developing improvements in (intelligent) collection and analysis capabilities [11,12]. Moreover, this is accompanied by numerous novel communication channels, which have led to a veritable "information explosion" [9,1,7]. Today's problem therefore no longer lies in acquiring information, but in processing its sheer volume [6]. In addition, technologies originally restricted to defense and intelligence services are now accessible to the general public, primarily via the Internet [7,12]. The understanding of intelligence thus changed completely [13]. At the same time, the increasing speed of development makes it almost impossible to predict the future shape of OSINT and its consequences [14].

To date, there has been a lack of decisive, fundamental scientific publications to pervade the opacity of the subject area [15] and address its rapid developments [vgl. 11,12]. In particular, there is a lack of current studies that reveal the actual technologies behind OSINT in detail and determine their characteristics. The question of whether "third generation" OSINT systems in the sense of robust, self-managing solutions [4,8] already exist has therefore not yet been clarified [11,8,9]. Moreover, the majority of studies focus exclusively on analyzing the OSINT trend area "cyber security" [cf. e.g.: 7,9,4]. The literature thus missed to cover the topic in its entirety. Important application scenarios ("use cases") of OSINT have therefore remained unconsidered in research to date [cf. also 16,13,11]. In addition, supplementary qualitative field research is absent, for example in the form of expert interviews, which contrast theory with the corresponding practical implementation. Although OSINT has a major impact on topics such as security and defense, there is a lack of insight into these sectors [4,15]. This paper hence aims to answer the research question:  \textit{How can the current trends in OSINT in the form of the technologies used and their characteristics, in particular the maturity level and the use case, be presented in a trend radar and validated by experts within the security sector?}

This paper aims to identify the technologies used in OSINT applications and to present them systematically in a trend radar, according to their characteristics. Through expert interviews, the identified trends will then be validated and compared with the common practical "reality". In this way, a well-founded knowledge base will be compiled, and existing research gaps of practical relevance will be identified. This will enable a coordinated exploration of the research field. The structure of this study thereby follows the iterative "Design Science Research Model" (DSRM) [17], an open research paradigm for the creation of an innovative artifact [18]. Within this framework, the relevant literature on OSINT will first be analyzed and classified using a systematic literature review [19]. Based on this, the OSINT technologies and their characteristics identified will be visualized in the form of a trend radar. The radar will then be validated using systematizing interviews [20] conducted with experts in the security sector. The interviews will then finally be evaluated using a qualitative content analysis [21].



\section{Formatting your paper}

All printed material, including text, illustrations, and charts, must be kept within a print area of 6-1/2 inches (16.51 cm) wide by 8-7/8 inches (22.51 cm) high. Do not write or print anything outside the print area. All text must be in a two-column format. Columns are to be 3 inches (7.85 cm) wide, with a 5.1/16 inch (0.81 cm) space between them. Text must be fully justified. \\
This formatting guideline provides the margins, placement, and print areas. If you hold it and your printed page up to the light, you can easily check your margins to see if your print area fits within the space allowed.

\section{Main title}

The main title (on the first page) should begin 1-3/8 inches (3.49 cm) from the top edge of the page, centered, and in Times 14-point, boldface type. Capitalize the first letter of nouns, pronouns, verbs, adjectives, and adverbs; do not capitalize articles, coordinate conjunctions, or prepositions (unless the title begins with such a word). Leave two 12-point blank lines after the title.

\section{Author name(s) and affiliation(s) }

Author names and affiliations must be included in the submitted Final Paper for Publication. Leave two 12-point blank lines after the author’s information.

\section{Second and following pages}
\label{sect:pdf}

The second and following pages should begin 1.0 inch (2.54 cm) from the top edge. On all pages, the bottom margin should be 1-1/8 inches (2.86 cm) from the bottom edge of the page for 8.5 x 11-inch paper. (Letter-size paper)

\section{Type-style and fonts}
\label{sec:type-style}

Please note that {\em Times New Roman} is the preferred font for the text of you paper. \textbf{If you must use another font}, the following are considered base fonts.  You are encouraged to limit your font selections to Helvetica, Arial, and Symbol as needed. These fonts are automatically installed with the viewing software.

\section{Page Numbers}

Please DO NOT include page numbers in your manuscript.



\section{Graphics/Images}

All images must be embedded in your document or included with your submission as individual source files. The type of graphics you include will affect the quality and size of your paper on the electronic document disc. In general, the use of vector graphics such as those produced by most presentation and drawing packages can be used without concern and is encouraged.

\begin{itemize}
    \item Resolution: 600 dpi
    \item Color Images: Bicubic Downsampling at 300dpi
    \item Compression for Color Images: JPEG/Medium Quality
    \item Grayscale Images: Bicubic Downsampling at 300dpi
    \item Compression for Grayscale Images: JPEG/Medium Quality
    \item Monochrome Images: Bicubic Downsampling at 600dpi
    \item Compression for Monochrome Images: CCITT Group 4
\end{itemize}

If your paper contains many large images they will be down-sampled to reduce their size during the conversion process.  However the automated process used will not always produce the best image, and you are encouraged to perform this yourself on an image by image basis. The use of bitmapped images such as those produced when a photograph is scanned requires significant storage space and must be used with care.

\section{Main text}

Type your main text in 10-point Times, single-spaced. Do not use double-spacing. All paragraphs should be indented 1/4 inch (approximately 0.5 cm).  Be sure your text is fully justified—that is, flush left and flush right. Please do not place any additional blank lines between paragraphs. \\
\textbf{Figure and table captions} should be 9-point boldface Helvetica (or a similar sans-serif font).  Callouts should be 9-point non-boldface Helvetica. Initially capitalize only the first word of each figure caption and table title. Figures and tables must be numbered separately. For example: ``Figure 1. Database contexts'', ``Table 1. Input data''. Figure captions are to be centered below the figures. Table titles are to be centered above the tables.

% For one-column wide figures use
\begin{figure}[thb]
    % Use the relevant command to insert your figure file.
    % For example, with the graphicx package use
    \centering
    \includegraphics[trim={3cm 3cm 3cm 3cm}, clip,width=0.9\linewidth]{sample-image}
    % figure caption is below the figure
    \caption{Sample figure with caption.}
    \label{fig: sample-figure}       % Give a unique label
\end{figure}

\section{First-order headings}

For example, “1. Introduction”, should be Times 12-point boldface, initially capitalized, flush left, with one 12-point blank line before, and one blank line after. Use a period (“.”) after the heading number, not a colon.

\subsection{Second-order headings}

As in this heading, they should be Times 11-point boldface, initially capitalized, flush left, with one blank line before, and one after.

\subsubsection{Third-order headings. }

Third-order headings, as in this paragraph, are discouraged. However, if you must use them, use 10-point Times, boldface, initially capitalized, flush left, followed by a period and your text on the same line.

\section{Footnotes}

Use footnotes sparingly and place them at the bottom of the column on the page on which they are referenced. Use Times New Roman 8-point type, single-spaced. To help your readers, try to avoid using footnotes altogether and include necessary peripheral observations in the text (within parentheses, if you prefer, as in this sentence).
asdöfj
% Fonts specification --- not shown as it doesn't exist in the Word document either. 

%\section{Fonts}

%A summary of fonts is provided in Table \ref{tab: fonts}. 

%\begin{table}[thb]
%\centering
%\caption{\label{font-table} Font guide. \vskip 3pt }
%\label{tab: fonts}
%\begin{tabular}{l|rl}
%\hline \bf Type of Text & \bf Font Size & \bf Style \\ \hline
%paper title & 14 pt &  \bf bold \\
%authors & 10 pt &  \underline{email} underlined \\
%abstract title & 12 pt &  \bf bold\\
%abstract text & 10 pt &  \it italic\\
%section titles & 12 pt & \bf bold \\
%subsection titles & 11 pt & \bf bold \\
%document text & 10 pt  & \\
%captions & 9 pt & \sansserifformat{\captionsize sans-serif, \bf bold} \\
%bibliography & 9 pt & \\
%footnotes & 8 pt & \\
%\hline
%\end{tabular}
%\end{table}


\section{References}

List and number all bibliographical references in 9-point Times, single-spaced, at the end of your paper. When referenced in the text, enclose the citation number in square brackets, for example \cite{Jones2015,Smith2015} and \cite{Smith2015}. Where appropriate, include the name(s) of editors of referenced books.

% if added before the last page, this command can help balancing columns
%\addtolength{\textheight}{-.2cm} 

%Bibliography 
\bibliographystyle{ieeetr}
\bibliography{sample}


\end{document}
