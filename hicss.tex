% File hicss51.tex
%%
%% Based on the style files for ACL 2015 by 
%% car@ir.hit.edu.cn, gdzhou@suda.edu.cn


\documentclass[10pt]{article}
\usepackage[letterpaper]{geometry}
\usepackage{hicss51}
\usepackage{times}
\usepackage[none]{hyphenat}
\usepackage{url}
\usepackage{latexsym}
%\usepackage{minted}
\usepackage{indentfirst}
\usepackage{graphicx}
%\graphicspath{{images/}}
\usepackage{wrapfig}
\usepackage{todonotes}
\usepackage{hyperref}
\usepackage[utf8]{inputenc}
\newcommand{\sansserifformat}[1]{\fontfamily{cmss}{ #1}}%

%\setlength\titlebox{5cm}


% You can expand the titlebox if you need extra space
% to show all the authors. Please do not make the title box
% smaller than 5cm (the original size).



\title{Open Source Intelligence - Development of a Trend Radar utilizing a Systematic Literature Review}

\author{Franz Kayser \\
  ESG \\
  {\underline{ franz.kayser@esg.de}} \\\And
  Thomas Mayer \\
  ESG  \\
  {\underline{ thomas3.mayer@esg.de} }\\\And 
  Michael Bücker \\
  FH Münster -- University of Applied Sciences\\
  {\underline{michael.buecker@fh-muenster.de}} \\}

\date{}

\begin{document}
\maketitle
\begin{abstract}


    Open Source Intelligence (OSINT) is currently experiencing an intensive discourse,
    heightened since the Russian invasion of Ukraine. However, despite numerous attempts
    at standardized definitions, the intelligence discipline remains ambiguous. This paper
    introduces a practice-validated OSINT trend radar, categorizing technologies by maturity,
    intelligence cycle phase, and use case. Serving as a profound knowledge base and tool for
    identifying research gaps, the radar emerges from a structured design process. Sixty
    studies underwent categorization and validation through expert interviews,
    revealing the absence of a comprehensive, autonomous third-generation OSINT
    system in Germany. Technological gaps, especially in the planning, direction,
    dissemination, and integration phases, are evident. Although intelligent support
    technologies were identified, practical implementation lags behind theory. The human
    factor therefore remains central to the OSINT process. Future research should thus
    prioritize developing applications for underserved phases, probing reasons for limited
    widespread implementation of proven applications, with emphasis on legal, ethical,
    political, and social parameters.


\end{abstract}

\section{Introduction}

OSINT is a currently more debated research field than ever before. Obtaining intelligence
from publicly available data \cite{DosPassos.2017} has become undeniably important since the
Russian invasion of Ukraine in 2022. The real-time analysis of social media in particular has
brought highly relevant insights to light \cite{Hatfield.2023, SmithBoyle.24.07.2023}. However, despite
numerous attempts to define OSINT (e.g. \cite{Hwang.2022, PastorGalindo.2020, Yogish.2021}),
the concept remains controversial to this day \cite{Ghioni.2023, Ish.2022,Williams.2018}.
This is not least due to the fact that every definition of OSINT is subject to the advances
in computer and data sciences, which continuously produce improvements in (intelligent)
collection and analysis possibilities \cite{Ghioni.2023, Williams.2018}. In addition, this is
accompanied by numerous new open means of communication, which have caused a veritable
"information explosion"\cite{DosPassos.2017, Hwang.2022, Yogish.2021}.
Data sources that were originally reserved for the defense and intelligence services are
now also publicly accessible \cite{Hwang.2022, Williams.2018}. The understanding of
intelligence thus changed completely \cite{Dokman.2020}.

To date, there has been a lack of fundamental scientific publications to penetrate
the subject area \cite{HerreraCubides.2020} and address its rapid
developments \cite{Ghioni.2023, Williams.2018}. In particular, there is a shortage of
studies that identify the technologies behind OSINT and determine their characteristics.
The question of whether autonomous third-generation OSINT systems \cite{PastorGalindo.2019, PastorGalindo.2020} exist
has not yet been clarified \cite{Ghioni.2023, PastorGalindo.2020,Yogish.2021}.
Moreover, the majority of studies focus exclusively on the OSINT trend area "cyber
security" \cite{Hwang.2022, PastorGalindo.2019, Yogish.2021}. The literature thus failed to
cover the topic in its entirety. Important use cases of OSINT have
therefore not yet been taken into account in research \cite{AlKilani.2021, Dokman.2020, Ghioni.2023}.
In addition, supplementary qualitative field research is absent to contrast theory with
the corresponding practical implementation. Although OSINT has a major impact on topics
such as security and defense, there is a lack of insight into these sectors \cite{HerreraCubides.2020, PastorGalindo.2019}.
This paper is hence dedicated to answering the research question:
\textit{How can the current trends in OSINT in the form of the technologies used and their
    characteristics, in particular the maturity level and the use case, be presented in a
    trend radar and validated by experts within the security sector?}

The aim is to identify the technologies used in OSINT applications and to present
them systematically in a trend radar, according to their characteristics. In this way, a well-founded knowledge base will
be compiled, and practice-relevant research gaps will be identified for a coordinated examination
of the research field. The study thereby follows the "Design Science Research Model" (DSRM)
\cite{Peffers.2007}.  First, the relevant literature on OSINT will be analyzed and classified
through a systematic literature review \cite{Webster.2002}. Second, the OSINT technologies and
their characteristics identified will be visualized in the form of a trend radar. The radar
will then be validated using systematizing interviews \cite{Bogner.2014} conducted with
experts in the security sector. Finally, the interviews will be evaluated using a
qualitative content analysis [21].

\section{Theoretical Background}

The domain of OSINT is continuously expanding due to the ongoing improvements of
collecting and analysis possibilities \cite{AlKilani.2021, Ghioni.2023, Williams.2018}. In
addition, the new means and methods of communication associated with advances in information
and communication technology have turned OSINT into a complex discipline
\cite{AlKilani.2021, Benes.2013, Chen.2012, Williams.2018}. OSINT and its
components are therefore first defined in detail hereinafter.

\subsection{Open Source Intelligence (OSINT) and its Components}

One of the earliest and still frequently referenced definitions \cite{DosPassos.2017}
was published by NATO in 2001. OSINT according to this definition is information that has been
deliberately discovered, discriminated, distilled, and disseminated to a select audience,
[...], in order to address a specific question. OSINT, [...] thus applies the proven
process of intelligence to the broad diversity of open sources [...] and creates
intelligence [23]. However, today the discipline is no longer seen as a purely governmental
matter. Private research institutions and organizations \cite{Bohm.2021,Mercado.2005} are
also massively driving the development of such systems, e.g. for competitive analyses or marketing activities
\cite{AlKilani.2021, Dokman.2020,Ghioni.2023}. The focus is thereby shifting to
developing OSINT into a robust, autonomous solution \cite{Billings.1997,PastorGalindo.2019,Schaurer.2010}.
The starting point for all OSINT activities therby lies in data. Data forms the basis of the
analysis and the conclusions derived from it \cite{Gibson.2016}. In this context, OSD
refers to non-processed \cite{DosPassos.2017}, general raw data that is openly available
\cite{Burke.2007} as well as legally and ethically accessible
\cite{Schaurer.2010, NorthAtlanticTreatyOrganization.2001}. Before intelligence can be obtained
from OSD, they must undergo a preparation process that includes filtering, validation and
summarization \cite{DosPassos.2017, NorthAtlanticTreatyOrganization.2001}. The result of this
data organization \cite{Schaurer.2010} is referred to as OSINF. It provides the basis for the
resulting intelligence creation \cite{DosPassos.2017,Schaurer.2010}.

\subsection{Intelligence and Intelligence Cycle}

The core task of OSINT is to generate intelligence \cite{Hwang.2022,Dokman.2020}
from the condensed information in terms of a profound basis for decision-making
\cite{Breakspear.2013,May.2020}. The generation process of such an intelligence product
is also referred to as the intelligence cycle \cite{HerreraCubides.2020, CentralIntelligenceAgency.1987}.
It represents the central element of every intelligence discipline \cite{Reuser.2017,Dokman.2020}.
The representation of the process as a cycle \cite{DirectorofNationalIntelligence.2011} dates
back to the CIA in 1987 \cite{CentralIntelligenceAgency.1987}. The link between the phases is that
the result of the preceding phase serves as input for the subsequent phase
\cite{JointChiefsofStaffU.S.Army.2013,Pellissier.2013}. Furthermore, the individual phases are also continuously
iterated due to the fulfillment of previous requirements and new demands \cite{Gibson.2016}.
Today, to represent external influences or the
assignment of responsibilities \cite{Lowenthal.2020,Phythian.2013,Johnston.2005}, numerous
variations can be found \cite{Bohm.2021,Reuser.2017}. The
Intelligence Cycle should therefore be seen less as a guideline and more as an informal
coordination element\cite{Hwang.2022}.
In an updated version from 2013, the JCS segmented the cycle into 6 phases [32] (see Fig \ref{fig: intelligence cycle}).

\begin{figure}[h]
    \centering
    \includegraphics[clip,width=0.9\linewidth]{Intelligence Cycle}
    \caption{Intelligence Cycle, according to \cite{JointChiefsofStaffU.S.Army.2013}}
    \label{fig: intelligence cycle}
\end{figure}

The planning and direction phase combines the identification, definition and prioritization
of the requirements for the cycle. It is also responsible for developing the activities
required to achieve these \cite{DepartmentoftheArmy.2012} and monitoring their implementation
\cite{JointChiefsofStaffU.S.Army.2013, DepartmentoftheArmy.2012}.
The collection phase refers to the gathering of raw data \cite{CentralIntelligenceAgency.1987}.
The core of this phase consists of the iterative repetition of research
\cite{NorthAtlanticTreatyOrganization.2001} to make the query more precise with each run
\cite{PastorGalindo.2020}. The processing and utilization phase involves condensing
these data volumes into valuable and action-relevant information
\cite{DirectorofNationalIntelligence.2011, JointChiefsofStaffU.S.Army.2013, PastorGalindo.2020}.
Analysis and production refers to the synthesis of the information obtained into a
user-oriented, timely and accurate intelligence product
\cite{DepartmentoftheArmy.2012, Hwang.2022, NorthAtlanticTreatyOrganization.2001}.
The final phase consists of handing over the finished product to the "customer" in a
usable form \cite{CentralIntelligenceAgency.2023, DepartmentoftheArmy.2012, Williams.2018}.
Evaluation and feedback are not to be regarded as individual phase within the cycle
but take place continuously throughout the entire process. The aim is to achieve progressive optimization
\cite{DirectorofNationalIntelligence.2011, JointChiefsofStaffU.S.Army.2013, NorthAtlanticTreatyOrganization.2001}.

\subsection{Previous Studies}

Eight previous, publicly accessible literature reviews can be identified
concerning OSINT. In 2017, Dos Passos \cite{DosPassos.2017} showed how big data and data science can
make the decision-making process more useful and effective. Pastor-Galindo et al. then described
the current state of OSINT in 2019 \cite{PastorGalindo.2019} and 2020 \cite{PastorGalindo.2020}
focusing on services and techniques to improve cybersecurity. Moreover, they are also responsible
for the first and only rudimentary mapping of OSINT trends. They observed that OSINT is used in
social opinion and sentiment analysis, cybercrime and organized crime, as well as cybersecurity and cyberdefence.
Two further literature reviews were published in 2020. García Lozano et al. \cite{GarciaLozano.2020} identified
methods for computer-assisted veracity assessment of public information.
Herrera-Cubides et al. \cite{HerreraCubides.2020} investigated how the production of
research and educational materials in the field of OSINT has developed. They concluded that
the number of publications is lower compared to other trending topics. In 2021,
Yogish and Krishna \cite{Yogish.2021} explored the state of implementation and use of
Artificial Intelligence (AI) technologies in the context of cybersecurity. The result of this
study showed that Machine Learning (ML), pattern recognition and Natural Language Processing
(NLP) can simplify OSINT in view of increasing data volumes. In the following year, Hwang et al.
\cite{Hwang.2022} investigated security threats and cybercriminality in the context of OSINT misuse.
In 2023, Ghioni et al. \cite{Ghioni.2023} then examined the political, ethical, legal and social implications of
OSINT in conjunction with AI. They discovered that there is still no framework
for addressing these. They also found that third-generation OSINT is still in its early
stages and that human components cannot yet be replaced.

\section{Research Methodology}

The structure of the study is based on the iterative DSRM \cite{Peffers.2007}. It is a theory-based
research paradigm for developing an explicitly applicable solution, in the form of an innovative artifact,
\cite{vomBrocke.2020b}, solving a (practical) problem \cite{Peffers.2007,Hevner.2004}. The model is therefore ideally suited to
create the trend radar. It consists of six successive activities (see Fig. \ref{fig: DSRM}) \cite{Peffers.2007}.
Moreover, the first three evaluation steps according to Sonnenberg and vom Brocke \cite{Sonnenberg.2012}
were applied to continuously evaluate the design process.

\begin{figure*}[thb]
    \centering
    \includegraphics[width=\textwidth]{DSRM}
    \caption{Design Science Research Model (DSRM)}
    \label{fig: DSRM}
\end{figure*}

\subsection{Problem Identification and Motivation}

First, the research problem must be defined and the benefits of the solution explained.
These activities can be found in the chapter Introduction.

\subsection{Design Objectives of the Solution}

The next step is to define the design objectives of the solution. The
objectives can be divided into two content-related objectives (CO) and
two formal objectives (FO). CO1 requires the trend radar to follow a procedural
structure that reflects the process of generating intelligence. This will allow a structured mapping of the
identified technologies according to their use. In doing so, any
research gaps that become apparent can be directly assigned to the
respective phase. It will thus be possible to verify if a third-generation OSINT system exists. As CO2, it was defined that the key
characteristics, in particular the maturity level of the technologies
and their use cases, must be taken into account. The maturity level of
the technologies makes it possible to determine the respective research status.
Through the use cases, the research directions can be
revealed. As FO1, it was determined that the trend radar must follow
a simple structure to enable the immediate identification of research gaps. In addition,
the radar should have a high degree of standardization to be
transferable to other intelligence gathering disciplines in later
studies. As FO2, it was decided that the trend radar should be continuously expandable to capture the high field dynamics.

\subsection{Evaluation of the Problem Statement and Design Objectives}

In the theoretical background section, the main definitions, in
particular, the intelligence cycle, which serves as a basis for the
development of the trend radar ("suitability"), were presented. In
addition, the presentation of previous research work showed that there
is still a lack of broad-based studies examining OSINT. Both the
relevance of the research question and the suitability of the
design objectives are thus underpinned \cite{Sonnenberg.2012}.

\subsection{Design and Development}

The third activity involves the creation of the artifact. For this
purpose, the relevant literature on OSINT was analyzed through a
systematic literature review \cite{Cleven.2009}. To clearly define the scope of the literature
review, Cooper's taxonomy \cite{Cooper.1988} was used. Subsequently, classification
categories were defined to establish concept matrices \cite{Webster.2002},
to systematically analyze the researched literature. To achieve a high degree of standardization,
general categories were defined for each phase of the intelligence
cycle. Thereby, the categories of the collection phase differ from
those of the remaining phases, as the data basis is also considered
here. The evaluation and feedback phase was not treated separately due
to its iterative nature.

The following six categories were defined for the collection
phase: "use case", "data", "process", "technology", "technology
complexity" and "Maturity level". First, the area of application
of the technologies was recorded under the use case. The
category data reveals the composition of the data foundation. Therefore,
it is further subdivided into the data type to record the formats and
the source to show the origin of the data. The category
process is used to determine the degree of automation of the technologies. For this purpose, the following
four levels were defined: manual, semi-automated, automated,
\cite{Duncheon.2002} fully automated/autonomous \cite{Billings.1997,Endsley.1999}.
To improve categorization, the ten levels of automation \cite{Sheridan.1978,Parasuraman.2000}
were additionally used. The fourth category serves to capture the technologies, defined
as "the totality of material and immaterial means available for
the input, output, conversion, transmission and storage of information" \cite{Bleck.2004}.
The fifth category evaluates the complexity of the technologies
examined. For the collection phase, the three subcategories
"Volume" \cite{OLeary.2012}, "Variety" and "Velocity" were used \cite{Elgendy.,Russom.2011,Singh.2012}.
Variety is further subdivided according to the data structure (structured \cite{Lin.2018},
semi-structured and unstructured \cite{Katal.2013,Praveen.2020}) \cite{OLeary.2012}. Velocity,
is further subdivided into the levels "Batch" \cite{Carbone.2015}, "Near Real-Time"
\cite{Stankovic.1990,Gomes.2021} and "Real-Time" \cite{Stankovic.1990}. The sixth category reflects the
maturity level of the technologies used, based on the three macro-maturity phases of the
German Federal Trend Radar: the innovation phase, the prototype phase and
the market establishment phase \cite{Stich.2022}.

For the remaining phases the categories use case, process,
technology, technology complexity and maturity level
were likewise defined. However, the complexity within the remaining
phases is measured based on the underlying analysis, in
ascending order: descriptive analysis, diagnostic analysis, predictive
analysis, prescriptive analysis \cite{Delen.2013,GartnerGmbH.2012}.

After defining the classification criteria, the literature research
was carried out (Fig. \ref{fig: Literature review}). A search string based on
general terms for the highest possible number of hits was determined.
The search results were delimited to publications in German or English. Only studies
with full access were considered. As the number of publications
increased significantly between 2020 and 2023, the period was reduced
to these years to ensure the most up-to-date coverage possible.
The studies were downloaded on 05.06.2023 (see Appendix D.B). The
SQR3 method \cite{Robinson.1970} was applied for the systematic analysis and further
delimitation. Altogether 60 studies were categorized following this
procedure.

\begin{figure*}[thb]
    \centering
    \includegraphics[width=\textwidth]{Literature review}
    \caption{Literature review}
    \label{fig: Literature review}
\end{figure*}

For categorization, a dedicated Excel spreadsheet was created per
Intelligence Cycle phase and the OSINT technologies and their
characteristics were methodically recorded \cite{Cleven.2009, Webster.2002}. The identified
technologies were then grouped according to relations in an
additional column (see Repository X). Afterward, the categorization was
verified using a "Python" script (see Appendix A). This
queries the occurrence of predefined keywords in the included papers.
Afterwards the trend radar was created based on the verified concept matrices.

\subsection{Evaluation of the Design Specifications}

The intelligence cycle as the basis of the trend radar provides a
structured overview ("clarity") \cite{Breakspear.2013}. This
fulfills the requirement of an intuitively understandable illustration
("understandability") for ensuring a simple extraction ("simplicity")
of technologies and research gaps. Furthermore, the other intelligence
disciplines are also derived from the cycle, which enables a later
application to these ("applicability"). By aligning the
architecture of the radar with the federal government's trend radar
\cite{Stich.2022}, a proven, robust, user-friendly
design ("user-friendliness") with an appropriate level of detail
("level of detail") is used. Only the categories use case, technology and
degree of maturity were therefore illustrated in the trend radar. Moreover, with the concept matrices, a
template is used to constantly update the radar. Particular attention
was paid to the general validity of the categories to achieve
the highest possible degree of standardization ("generality").
Furthermore, a verification of the categorization was carried out ("internal consistency").
The decisive evaluation criteria \cite{vomBrocke.2020b} are thus fulfilled.

\subsection{Demonstration}

Next, the trend radar was demonstrated via guideline-based,
systematizing expert interviews \cite{Bogner.2014, Glaser.2009, Meuser.1991}.
Especially in less structured and sparsely linked subject areas, the method
enables dense data collection \cite{Bogner.2014,Meuser.1991}. Additionally,
it is suitable in cases where access to the social field is limited \cite{Bogner.2002c, Glaser.2009}.

The experts (Tab. \reference{tab: experts}) were selected according to the method of
theoretical sampling \cite{Glaser.1967,Eisenhardt.1989}. It was specified that only experts in
Germany should be interviewed to validate the trends in this
country. It was also determined that at least one expert each
from a security authority, the security industry and a
start-up would be selected to capture different points of
view. A "prestigious" company position is seen as a reliable
guarantee that the respondents possess research-relevant knowledge \cite{Bogner.2002b}.

\begin{table}[htbp]
    \label{tab:experts}
    \caption{Interviewed experts}
    \begin{tabular}{|p{0.25\linewidth}|p{0.55\linewidth}|p{0.05\linewidth}|}
        \hline
        \textbf{Organization} & \textbf{Position}                                                         & \textbf{ID} \\
        \hline
        Industry/ Authority   & Senior Intelligence Consultant                                            & E1          \\
        \hline
        Industry/ Authority   & Referent Corporate Security                                               & E2          \\
        \hline
        Authority             & In-House Senior Consultant                                                & E3          \\
        \hline
        Start-up              & Co-Founder, Managing Director of a German start-up, for an OSINT platform & E4          \\
        \hline
    \end{tabular}
\end{table}

The qualitative data collection that followed was carried out using
semi-structured interviews. These are particularly suitable for
revealing the underlying relationships of a theory \cite{Bogner.2014}. The
questionnaire used is based on the structure of the Intelligence Cycle.
At the beginning of the interview, the trend radar was presented.
To compare it with the respondents' practical experience, open questions
were asked for each phase. These reduce the influence of
subjectivity \cite{Saunders.2012}. Exploratory questions were added to direct the flow of conversation
\cite{Saunders.2012}. At the third level, specific closed questions for targeted follow-up
queries were asked \cite{Saunders.2012}. The interviews lasted up to one hour, with a maximum of three
main questions for each phase \cite{Bogner.2014} (see Respository X). The questionnaire
was pilot-tested with a domain expert. The expert interviews were conducted online via
platforms with video chat functions.

\subsection{Evaluation of the First Instance of the Trend Radar}

The demonstration of the trend radar confirmed its intuitive
comprehensibility ("ease of use"). It was also perceived by the
practitioners as a useful tool for providing an overview of OSINT
technologies ("effectiveness"). In addition, they confirmed its
completeness ("completeness") and internal consistency ("consistency")
(cf. expert E1, 14.08.2023; expert E3, 28.07.2023; expert E4,
02.08.2023). The trend radar thus proved to be a
suitable instrument for identifying research gaps in theory-practice
comparisons and for serving as a guideline to practitioners
("fidelity with real world phenomenon"). The essential evaluation
criteria \cite{Sonnenberg.2012} are thus demonstrably fulfilled.

\subsection{Evaluation}

The evaluation was carried out using a qualitative data analysis \cite{Glaser.2009}.
The aim of this is to extract,
synthesize and structure the information contained in the interviews
using a predefined search grid. This enables the targeted
extraction and summarization of relevant, cross-interview information
according to a "top-down approach" \cite{Bogner.2014, Glaser.2009}.

First, the recorded interviews were transcribed. Second, the software
MAXQDA \cite{MAXQDA.19.07.2023} was used for the subsequent categorization and analysis.
The categorization system was therefore established within MAXQDA
(see repository X). The categories of the first level correspond to
the individual phases of the intelligence cycle. The categories of the
second level allow a classification of whether the experts express
themselves in support of or in contradiction to the theory. The
categories of the third level reflect the identified use cases.
As many of the experts' statements were general, the
category "General statements" was added. The categories
of the fourth level reflect the individual technologies classified
under them. Altogether, 257 statements were categorized in this way. Examples of the coding
procedure can be found in (Tab. \ref{tab:coding}).

\begin{table*}[htbp]
    \caption{Coding examples}
    \label{tab:coding}
    \begin{tabular*}{\textwidth}{|p{0.12\linewidth}|p{0.12\linewidth}|p{0.12\linewidth}|p{0.12\linewidth}|p{0.39\linewidth}|}
        \hline
        \textbf{Level 1} & \textbf{Level 2} & \textbf{Level 3} & \textbf{Level 4} & \textbf{Transcript example} \\
        \hline
        Collection phase & Theory- supporting & General statement & "web crawler" and/or "web scraper" & And precisely because there are so many, so simple ways to create web crawlers and web scrapers, [...]"(Cf. expert E1, 14.07.2023) \\
        \hline
        Analysis and production phase & Theory- contradictory & Security & AI, ML, DL & "Deep learning, machine learning, artificial intelligence, in some places, I don't know anyone who has built in a random forest anywhere [...]" (cf. E3, 02.08.2023) \\
        \hline
    \end{tabular*}
\end{table*}

\subsection{Communication}

The research results are communicated in the form of this study.
A new iteration, proceeding from this chapter, is therefore not
carried out.

\section{Results}

The trend radar (see Fig. X) is read from the outermost to the innermost.
Each fifth of the cycle represents an Intelligence Cycle phase. Subdivisions indicate
phase-specific use cases, while color gradations show maturity levels. Numbered black
and white dots denote grouped technologies, presented in a boxplot-like format reflecting
varying maturity levels.

\subsection{Intelligence Cycle in Theory-Practice Comparison}

Studies could be attributed to each of the Intelligence Cycle phases. However,
none of the applications identified covers all phases in the sense of a third-generation
OSINT tool. Literature primarily focuses on the collection phase, followed by
the analysis and production phase and the processing and exploitation phase.
The dissemination and integration phase is covered by far the least after the planning
and direction phase.

These findings align with the experts' practical experiences. They regard the Intelligence Cycle
as "state of the art" (cf. E3, 28.12.2023), but note different manifestations of the phases in praxis (cf. E4, 02.08.2023).
The planning and direction phase is often neglected, despite its crucial importance, leading to wasteful
production (cf. E3, 28.07.2023). Conversely, OSINT is frequently associated solely with the collection phase,
resulting in subpar outcomes due to high volumes of low-quality data (cf. E1, 14.07.2023; E2, 19.07.2023; E3, 28.07.2023).
The reason for this is, among other things, that the Intelligence Cycle is operated by at least three groups of people.
Firstly, the product customers, usually located at the "decision-maker level", with a primarily legal professional background (cf. E2, 19.07.2023).
The second is the technician who carries out the data collection and processing (cf. E2, 19.07.2023; E3, 28.07.2023).
Lastly, the analyst who evaluates the data and creates the intelligence product
(cf. E1, 14.07.2023). The process thereby is rarely transparent between the parties
(cf. E1, 14.07.2023; E2, 19.07.2023) and is rarely anchored at the organizational level
(cf. E4, 02.08.2023). According to the experts, because of that there is no third-generation
OSINT tool in use, at least not in German authorities. In addition, the collection focus is driven by concerns
about missing vital information, which could later be revealed as publicly available (cf. E1, 14.07.2023).

\subsection{Use Cases in Theory-Practice Comparison}

Five main use cases emerged from the research: Cyber Security, Health, Security, Journalism,
and Competition Analysis. Cyber Security studies primarily focus on Open Source Cyber Threat
Intelligence (OSCTI), which involves collecting, monitoring, and analyzing publicly available
data to detect potential cyber threats (Ahuja et al., 2022, p.1; Al-Dmour et al., 2023, p. 1 f.).
Health applications mainly pertain to COVID-19, such as investigating the epidemic outbreak
(Kpozehouen et al., 2020; Thamtono et al., 2021). The security use case includes applications such as
analyzing violent behavior in public transport (Nobili et al., 2021). The identified Journalism study examines the
Twitter activities of the OSINT journalists' association "Bellingcat" (Bär et al., 2023). Competitive analysis
involves for example the performance classification of Chinese logistics companies (Tao et al., 2023). Additionally, two
studies could be identified on general approaches to creating knowledge graphs (Hu et al., 2023; Ma et al., 2022)

According to the experts, OSINT is applied in all authorities and has proven itself
in numerous use cases, even if not always explicitly labeled as such (cf. E2, 19.07.2023).
Most common cases are in cyber security/CTI (cf. E1, 14.07.2023; E2, 19.07.2023), as well as general security,
particularly with regard to the German Armed Forces, the (Federal) Intelligence Service,
the Ger­man do­mes­tic in­tel­li­gence ser­vices and the police.

\subsection{Technologies and Maturity Levels in Theory-Practice Comparison}

Except for the initial phase, automated technologies are utilized across all subsequent phases and use cases.
These technologies demonstrate considerable market maturity, yet manual activities remain prevalent.
The highest level of automation is observed in the CTI use case.

The most advanced automated technology in the collection phase is web crawlers and/or web scrapers.
Established technologies include "off the shelf" tools (cf. Middleton et al., 2020, p. 84) and open source
solutions like "Tweepy," a Python library for Twitter crawlers (cf. Adewopo et al., 2020, p. 2238; Tekin & Yilmaz, 2021, p. 84).
More advanced prototypes involve combining parallelized, recursive, source-specific web crawlers and scrapers for improved and
faster data collection (cf. Jenkins et al., 2021, p. 2 f.; Rajendran et al., 2022, p. 3 f.). Another method in the prototype phase is
"focused crawling," adapting the crawling path dynamically using a content-driven ML algorithm, BERT („Bidirectional Encoder Representation from Transformers“)
(cf. Kuehn et al., 2023, p. 1). Technologies for crawling/scraping the dark web, like "Torsion" (cf. Sonawane et al., 2022, p. 3),
were assigned to the innovation phase. The experts also note increasing use of open source tools alongside manual work
(cf. E1, 14.07.2023; E3, 28.07.2023; E1, 14.07.2023). However, they consider traditional web crawling and scraping
outdated due to errors, implementation difficulties, and website resistance. Screenshot-based "web shooting" with
subsequent OCR (Optical Character Recognition) extraction is seen as more modern and robust (cf. E3, 28.07.2023).

NLP applications/methods such as "Topic Classifying," "Part-of-Speech Tagging", "Entity and Relation Annotation", and "Named Entity Recognition"
demonstrate high automation levels in the processing and utilization phase. Technologies commonly used include
the "Python NLTK Toolkit" (Hubbard et al., 2022, p. 92 f.) and the "Stanford CoreNLP Toolkit" (Middleton et al., 2020, p. 85 f.).
Additionally, deep learning, particularly through "word embedding" using the "word2vec" algorithm, is prominent
(Bai et al., 2020, p. 459; Pingle et al., 2020, p. 882; Shen & Chow, 2020, p. 33 f.). However, they note that
this phase within authorities predominantly involves manual work due to the irreplaceable domain knowledge of experts
(cf. E1, 14.07.2023; E2, 19.07.2023; E3, 28.07.2023). The degree of automation of technologies thus depends on the abstraction level.
Operational tasks, which often require specific individual information, show lower automation levels compared to long-term general strategic analyses
requiring extensive data (cf. E3, 28.07.2023).

The highest automation level is observed in the analysis and production phase, where AI, ML, and DL technologies are prevalent.
Under DL, vectorization algorithms can also be found. Particularly BERT algorithms in different versions are commonly utilized
(cf. Ma et al., 2022, p. 3; Chen et al., 2023, p. 43 f.). Furthermore, even under ML vectorization models such as BERT or
"Supervised Support Vector Machines" (SVM) (see e.g.: Iorga et al., 2020, p. 4) are listed in the literature.
In addition, the algorithms "Random Forest", XGBoost ("eXtreme Gradient Boosting)",
lightGBM ("light Gradient Boosting Machine"), "Naive Bayes" and "logistic regressions" are particularly common.
Publications thereby often utilize multiple algorithms concurrently for performance comparison (e.g. Tao et al. (2023, p. 134 f.)
or for layered analysis (e.g. Yang et al. (2022, p. 19 f.). AI technologies are generally less specified,
except for cases like Dale et al. (2023, p. 436, 438), who developed a bidirectional recurrent neural network with
BiGur ("Bidirectional Gated Recurrent Unit") layers. Due to the modular use of publicly available models,
the technologies are mostly classified as market-ready. Despite the potential, practical work in this phase
largely relies on manual content analysis due to a lack of technological understanding and acceptance,
especially at the contractor/provider level in Germany (cf. E2, 19.07.2023; E3, 28.07.2023; E4, 02.08.2023). Furthermore,
ethical and legal barriers, such as GDPR (General Data Protection Regulation), also hinder technology adoption
(cf. E2, 19.07.2023; E4, 02.08.2023). Additionally, security concerns in German authorities favoring
standalone systems (cf. E2, 19.07.2023). However, there's a need for modular, dynamically expandable
systems to keep pace with rapid advancements (cf. E1, 14.07.2023; E3, 28.07.2023; E4, 02.08.2023).
If smart technologies are nevertheless used, it is often only inofficially, under the "desk" (cf. E4, 02.08.2023).
Nevertheless, human experience and specialization should not be outweighed, but rather a certain product quality
be ensured in a supportive manner. Yet, the revolutionary potential of large language models (LLM)
in the innovation phase cannot be estimated (cf. expert E4, 02.08.2023).


In the dissemination and integration phase, tools like "Power BI" (Tao et al., 2023, p. 134)
are utilized to create dashboards and visualization maps. arious user interfaces,
web applications, and online platforms are developed, including Python GUIs,
specific browser applications (Elmas et al., 2022, p. 392),
improved user interfaces and input masks for entire tool stacks (Arjun et al., 2020, p. 4). 
Additionally, Technologies for generating automated alerts, particularly for cyber security risk assessments, 
are prevalent (Ahuja et al., 2022, p. 1460 f.). Graph-based visualizations are also common, utilizing tools
/libraries like "Matplot," "Networkx," "Pygraphistry," or the "Neo4j-Browser" (Middleton et al., 2020, p. 85 f.).
Except from the alerts, the retrieval of results is largely semi-automated and the technologies are in the 
market establishment phase. No information on targeted user tests or a new development run involving 
user feedback could be found in any of the studies. The experts state that there is still very little 
automation within the authorities during this phase. The final product is often only a PDF document, 
an email or a verbal report (cf. E1, 14.07.2023), although in many cases no more is required (cf. E3, 28.07.2023). 
However, outside of the OSINT topic, there are several automated tools that could also be transferred 
to the authorities in this context (cf. E4 02.08.2023). Moreover, in practice there is also a lack of 
necessary feedback for product improvement (cf. E3, 28.07.2023).



\section{Graphics/Images}

All images must be embedded in your document or included with your submission as individual source files. The type of graphics you include will affect the quality and size of your paper on the electronic document disc. In general, the use of vector graphics such as those produced by most presentation and drawing packages can be used without concern and is encouraged.

\begin{itemize}
    \item Resolution: 600 dpi
    \item Color Images: Bicubic Downsampling at 300dpi
    \item Compression for Color Images: JPEG/Medium Quality
    \item Grayscale Images: Bicubic Downsampling at 300dpi
    \item Compression for Grayscale Images: JPEG/Medium Quality
    \item Monochrome Images: Bicubic Downsampling at 600dpi
    \item Compression for Monochrome Images: CCITT Group 4
\end{itemize}

If your paper contains many large images they will be down-sampled to reduce their size during the conversion process.  However, the automated process used will not always produce the best image, and you are encouraged to perform this yourself on an image by image basis. The use of bitmapped images such as those produced when a photograph is scanned requires significant storage space and must be used with care.

\section{Main text}

Type your main text in 10-point Times, single-spaced. Do not use double-spacing. All paragraphs should be indented 1/4 inch (approximately 0.5 cm).  Be sure your text is fully justified—that is, flush left and flush right. Please do not place any additional blank lines between paragraphs. \\
\textbf{Figure and table captions} should be 9-point boldface Helvetica (or a similar sans-serif font).  Callouts should be 9-point non-boldface Helvetica. Initially capitalize only the first word of each figure caption and table title. Figures and tables must be numbered separately. For example: ``Figure 1. Database contexts'', ``Table 1. Input data''. Figure captions are to be centered below the figures. Table titles are to be centered above the tables.

% For one-column wide figures use
\begin{figure}[thb]
    % Use the relevant command to insert your figure file.
    % For example, with the graphicx package use
    \centering
    \includegraphics[trim={3cm 3cm 3cm 3cm}, clip,width=0.9\linewidth]{sample-image}
    % figure caption is below the figure
    \caption{Sample figure with caption.}
    \label{fig: sample-figure}       % Give a unique label
\end{figure}

\section{First-order headings}

For example, “1. Introduction”, should be Times 12-point boldface, initially capitalized, flush left, with one 12-point blank line before, and one blank line after. Use a period (“.”) after the heading number, not a colon.

\subsection{Second-order headings}

As in this heading, they should be Times 11-point boldface, initially capitalized, flush left, with one blank line before, and one after.

\subsubsection{Third-order headings. }

Third-order headings, as in this paragraph, are discouraged. However, if you must use them, use 10-point Times, boldface, initially capitalized, flush left, followed by a period and your text on the same line.

\section{Footnotes}

Use footnotes sparingly and place them at the bottom of the column on the page on which they are referenced. Use Times New Roman 8-point type, single-spaced. To help your readers, try to avoid using footnotes altogether and include necessary peripheral observations in the text (within parentheses, if you prefer, as in this sentence).
asdöfj
% Fonts specification --- not shown as it doesn't exist in the Word document either. 

%\section{Fonts}

%A summary of fonts is provided in Table \ref{tab: fonts}. 

%\begin{table}[thb]
%\centering
%\caption{\label{font-table} Font guide. \vskip 3pt }
%\label{tab: fonts}
%\begin{tabular}{l|rl}
%\hline \bf Type of Text & \bf Font Size & \bf Style \\ \hline
%paper title & 14 pt &  \bf bold \\
%authors & 10 pt &  \underline{email} underlined \\
%abstract title & 12 pt &  \bf bold\\
%abstract text & 10 pt &  \it italic\\
%section titles & 12 pt & \bf bold \\
%subsection titles & 11 pt & \bf bold \\
%document text & 10 pt  & \\
%captions & 9 pt & \sansserifformat{\captionsize sans-serif, \bf bold} \\
%bibliography & 9 pt & \\
%footnotes & 8 pt & \\
%\hline
%\end{tabular}
%\end{table}


\section{References}


%Bibliography 

\bibliographystyle{ieeetr}
\bibliography{references}




\end{document}
